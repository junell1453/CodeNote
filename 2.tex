\documentclass{article}
\usepackage[UTF8]{ctex}
\usepackage{circuitikz}
\usetikzlibrary{positioning}

\begin{document}

\begin{figure}[h]
\centering
\begin{circuitikz}[scale=1.2, transform shape]
    % 定义输入节点位置
    \node (inputA) at (0,2) {输入A};
    \node (inputB) at (0,0) {输入B};
    
    % 绘制逻辑门
    \node[or port] (orGate) at (3,1.5) {};
    \node[nand port] (nandGate) at (3,-0.5) {};
    
    % 从输入A连接到两个门
    \draw (inputA.east) -- (orGate.in 1);
    \draw (inputA.east) |- (nandGate.in 1);
    
    % 从输入B连接到连接点,然后分叉到两个门
    \draw (inputB.east) -- ++(1,0) coordinate (junction) 
        -- (orGate.in 2);
    \draw (junction) -- (nandGate.in 2);
    
    % 标记连接点
    \filldraw (junction) circle (1.5pt);
    
    % 添加输出标签
    \draw (orGate.out) -- ++(0.5,0) node[right] {或门输出};
    \draw (nandGate.out) -- ++(0.5,0) node[right] {与非门输出};
    
    % 可选:添加门类型标注
    \node[above] at (orGate.south) {或门};
    \node[below] at (nandGate.north) {与非门};
    
\end{circuitikz}
\caption{包含或门和与非门的数字电路}
\end{figure}


\begin{figure}[h]
\centering
\begin{circuitikz}[scale=1.2, transform shape]
    \draw (0,0) node[left](A){$A$} to[short, o-]++(1,0)coordinate(JA)
    to[short,*-]++(.5,0) node[or port,anchor=in 1](or1){}
    (or1.in 2-|A.east) node[left]{$B$} to[short,o-]++(.5,0)coordinate(JB)
    to[short,*-](or1.in 2)
    (or1.out)-|+(.5,-.6)node[and port,anchor=in 1](and1){}
    (and1.in 2)|-++(-.5,-.6)node[nand port,anchor=out](nand1){}
    (JA)|-(nand1.in 1)
    (JB)|-(nand1.in 2);
\end{circuitikz}
\caption{包含或门和与非门的数字电路}
\end{figure}

\end{document}