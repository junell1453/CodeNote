\documentclass{standalone}
\usepackage{circuitikz} % 电路元件(电池、灯泡、导线等)
\usetikzlibrary{shapes.gates.logic.US, positioning} % 逻辑门形状与定位
\begin{document}

\begin{circuitikz}[american, line width=0.8pt]
% ==== 两路电源(电池符号) ====
% 第一路电源 V_A(上)
\draw (0,2) to[battery1,l=$V_A$] (0,0);
% 第二路电源 V_B(下)
\draw (0,-1.5) to[battery1,l=$V_B$] (0,-3.5);

% 把电源引出到 NOR 门的输入点
\draw (0,1) -- (1.6,1); % V_A 引线
\draw (0,-2.5) -- (1.6,-2.5); % V_B 引线

% ==== NOR 门(tikz 逻辑门) ====
% 我们把 NOR 门放在 (3.6,-0.75) 位置,两个输入分别连接到 gate.input 1/2
\node[nor gate US, draw, rotate=0, logic gate inputs=nn, minimum width=20mm] (G) at (3.6,-0.75) {};

% 将引线连接到 NOR 的输入引脚(使用 gate 的预定义锚点)
\draw (1.6,1) -- (G.input 1);
\draw (1.6,-2.5) -- (G.input 2);

% 在输入线附近加上小圆点表示连接点(可选)
\fill (1.6,1) circle (0.6pt);
\fill (1.6,-2.5) circle (0.6pt);

% 给两个输入标注高/低电平(可选示例)
\node[left] at (0,1) {Input A};
\node[left] at (0,-2.5) {Input B};

% ==== 输出到灯泡 ====
% 从 NOR 输出画导线到灯泡
% 把灯泡放在右侧(使用 circuitikz 的 lamp 元件)
\draw (G.output) -- ++(1.0,0) coordinate (toLamp)
to[bulb,l=$L$] ++(2.0,0) coordinate (lampRight);

% 从灯泡右端接回公共地(或回到电源负极,形成回路)
% \draw (lampRight) -- ++(0,-1.8) coordinate (downToGround);
% \draw (downToGround) -- (-0.5,-1.8) -- (-0.5,-3.5); % 回到底部电源负端附近
% % 标注地/负极连接(可根据需要修改为 GND 符号)
% \node[right] at (downToGround) {Return / Negative};

\draw (lampRight) node[ground]{};
% ==== 在 NOR 上标注(可选) ====
\node[above right=2pt of G] {NOR};

\end{circuitikz}

\end{document}