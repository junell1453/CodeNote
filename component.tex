\documentclass{article}
\usepackage[UTF8]{ctex}
\usepackage{circuitikz}

\begin{document}
\newcommand*{\ha}[2]{ %#1 node name,#2 anchor
    node[fourport,anchor=#2,scale=.65] (#1) {}
    (#1.port1) node[right]{B}
    (#1.port2) node[left]{CO}
    (#1.port3) node[left]{S}
    (#1.port4) node[right]{A}
}

\begin{circuitikz}  
    \ctikzset{quadpoles/fourport/width=2}
    \draw (0,0)node[left](A){A}to[short,o-]++(.5,0)
    \ha{A0}{port4}
    (A0.port3)--++(1,0)\ha{A1}{port1}
    (A0.port1)to[short,-o](A0.port1-|A.east)node[left]{B}
    (A1.port4)to[short,-o](A1.port4-|A.east)node[left]{Carry}
    (A1.port2)--++(.5,0)node[or port,anchor=in 1](or1){}
    (A0.port2)--(or1.in 2)
    (or1.out)node[right]{Carry}
    (A1.port3)--(A1.port3-|or1.out)node[right]{Sum};
\end{circuitikz}

\tikzset{fadder/.style={flipflop, flipflop def={
    t1=CI, t2=A, t3=B, t4=S,t6=CO}}
}
\begin{circuitikz}
    \ctikzset{multipoles/flipflop/width=1.5};
    \draw(0,0)node[fadder,scale=1.2]{};
\end{circuitikz}

\end{document}
