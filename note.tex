\documentclass{article}
\usepackage{xeCJK}                   % ✅ 中文支持
\setCJKmainfont{SimSun}              % ✅ 设置中文字体
\usepackage{circuitikz}
\tikzset{
  snot port/.style = {not port, scale=0.6}
}

\title{Code 笔记}
\author{Sunny}
\date{\today}
\begin{document}

\maketitle

\section{引言}
本文按Turing Complete游戏过关顺序,采用布尔代数表达。

\subsection{NAND 全阳转阴}
天地混沌,NAND为始。不知何人如何创建NAND门,然其功用至大,盖因其可构造任意逻辑电路也。
这种性质有个专属名词叫函数完备,俗称万能电路,只有NAND和NOR有这个性质。

\begin{center}
\begin{circuitikz}
  \draw
  (0,0) node[nand port] (nand1) {\texttt{NAND}}
  (nand1.in 1) node[left] {$A$}
  (nand1.in 2) node[left] {$B$}
  (nand1.out) node[right] {$\overline{A \cdot B}$};
\end{circuitikz}
\end{center}

\subsection{NOT逆阴阳}
\begin{circuitikz}
  \draw (0,0) node[above](A){$A$} to[short, o-] ++(1,0)
  node[nand port,anchor=in 1] (nand1){}
  (nand1.in 2)-|(A) 
  (nand1.out) node[right] {$A\  \texttt{NAND}\  A = \overline{A}$};

  \draw[shift={(8,0)}]
  (0,0) node[not port] (not1) {\texttt{N}}
  (not1.in 1) node[left] {$A$}
  (not1.out) node[right] {$\overline{A}$};  
\end{circuitikz}

\subsection{AND 全阳则阳}
\begin{circuitikz}
  \draw (0,0) node[nand port] (nand1) {}
  (nand1.in 1) node[left] {$A$}
  (nand1.in 2) node[left] {$B$}
  (nand1.out) --++(.2,0)
  node[snot port,anchor=in] (not1){}
  (not1.out)node[right]{$\overline{A\  \texttt{NAND}\  B} = A \cdot B$};

  \draw[shift={(7,0)}]
  (0,0) node[and port] (and1) {\texttt{AND}}
  (and1.in 1) node[left] {$A$}
  (and1.in 2) node[left] {$B$}
  (and1.out) node[right] {$A \cdot B$};
\end{circuitikz}

\subsection{OR 全阴则阴}
\begin{circuitikz}
  \draw (0,0) node[nand port] (nand1) {}
  (nand1.in 1)--++(-.3,0)node[not port,anchor=out,scale=.4](not1){}
  (nand1.in 2)--++(-.3,0)node[not port,anchor=out,scale=.4](not2){}
  (not1.in) node[left]{$A$}
  (not2.in) node[left]{$B$}
  (nand1.out)node[right] {$A\  \texttt{OR}\  B = A+B$};  

  \draw[shift={(6,0)}]
  (0,0) node[or port] (or1) {\texttt{OR}}
  (or1.in 1) node[left] {$A$}
  (or1.in 2) node[left] {$B$}
  (or1.out) node[right] {$A+B$};  

\end{circuitikz}

\subsection{NOR 全阴转阳}
\begin{circuitikz}[american]
  \draw (0,0) node[nand port] (nand1) {}
  (nand1.in 1)--++(-.3,0)node[not port,anchor=out,scale=.4](not1){}
  (nand1.in 2)--++(-.3,0)node[not port,anchor=out,scale=.4](not2){}
  (not1.in) node[left]{$A$}
  (not2.in) node[left]{$B$}
  (nand1.out)--++(.2,0)node[snot port,anchor=in](not3){}
  (not3.out)node[right] {$A\  \texttt{NOR}\  B = \overline{A} \cdot \overline{B}=\overline{A+B}$};

\end{circuitikz}

\begin{circuitikz}
  \draw
  (0,0) node[nor port] (nor1) {\texttt{NOR}}
  (nor1.in 1) node[left] {$A$}
  (nor1.in 2) node[left] {$B$}
  (nor1.out) node[right] {$\overline{A} \cdot \overline{B}=\overline{A+B}$};  
\end{circuitikz}

\subsection{XOR}
\begin{circuitikz}
  \draw (0,0) node[left]{$A$} to[short, o-]++(.5,0)coordinate(JA)
  (JA)--++(.5,0)node[snot port,anchor=in](nota){}
  (nota.out)-|++(.5,0) node[and port,anchor=in 1](and1){}
  (and1.out)-|++(.5,-.4)node[or port,anchor=in 1](or1){}
  (or1.in 2)|-++(-.5,-.4)node[and port,anchor=out](and2){}
  (and2.in 2)--++(-.5,0)node[snot port,anchor=out](notb){}
  (notb.in)--++(-.6,0)coordinate(JB)
  to[short,*-](JB|-and1.in 2)--(and1.in 2)
  (JB)--++(-.4,0)node[left]{$B$}node[ocirc]{}
  (JA)to[short,*-](JA|-and2.in 1)--(and2.in 1)
  (or1.out)node[right] 
    {$A\  \texttt{XOR}\  B=\overline{A} \cdot B+A\cdot \overline{B}$};    
\end{circuitikz}

\begin{circuitikz}
  \draw (0,0) node[xor port] (xor1) {}
  (xor1.in 1) node[left] {$A$}
  (xor1.in 2) node[left] {$B$}
  (xor1.out) node[right] {$A\oplus B$};  
\end{circuitikz}  

\end{document}
