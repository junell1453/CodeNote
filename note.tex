\documentclass{article}
\usepackage{xeCJK}                   % ✅ 中文支持
\setCJKmainfont{SimSun}              % ✅ 设置中文字体
\usepackage{circuitikz}

\title{Code 笔记}
\author{Sunny}
\date{\today}
\begin{document}

\maketitle

\section{引言}
NAND生万物。
天地混沌,NAND为始。不知何人如何创建NAND门,然其功用至大,盖因其可构造任意逻辑电路也。

\subsection{初识NAND}
\begin{circuitikz}
  \draw
  (0,0) node[nand port,
    label={[shift={(-.6,-0.5)}]below:{\textbf{NAND}}}
  ] (nand1) {}
  (nand1.in 1) node[left] {$A$}
  (nand1.in 2) node[left] {$B$}
  (nand1.out) node[right] {$Y = \overline{A \cdot B}$};
\end{circuitikz}


\subsection{加法初成}
\begin{circuitikz}[american]
  \draw
  (0,0) node[nor port,
    label={[shift={(-.6,-0.5)}]below:{\textbf{NOR}}}
  ] (nor1) {}
  (nor1.in 1) node[left] {$A$}
  (nor1.in 2) node[left] {$B$}
  (nor1.out) node[right] {$Y = A + B$};
\end{circuitikz}
\end{document}
