\documentclass{article}
\usepackage{xeCJK}                   % ✅ 中文支持
\setCJKmainfont{SimSun}              % ✅ 设置中文字体
\usepackage{circuitikz}

\title{Code 笔记}
\author{Sunny}
\date{\today}
\begin{document}

\maketitle

\section{引言}
本文按Turing Complete游戏过关顺序,采用布尔代数表达。

\subsection{NAND生万物}
天地混沌,NAND为始。不知何人如何创建NAND门,然其功用至大,盖因其可构造任意逻辑电路也。
这种性质有个专属名词叫函数完备,俗称万能电路,只有NAND和NOR有这个性质。

\begin{center}
\begin{circuitikz}
  \draw
  (0,0) node[nand port] (nand1) {\texttt{NAND}}
  (nand1.in 1) node[left] {$A$}
  (nand1.in 2) node[left] {$B$}
  (nand1.out) node[right] {$\overline{A \cdot B}$};
\end{circuitikz}
\end{center}

\subsection{NOT逆阴阳}
\begin{circuitikz}
  \draw (0,0) node[above](A){$A$} to[short, o-] ++(1,0)
  node[nand port,anchor=in 1] (nand1){}
  (nand1.in 2)-|(A) 
  (nand1.out) node[right] {$A\  \texttt{NAND}\  A = \overline{A}$};

  \draw[shift={(8,0)}]
  (0,0) node[not port] (not1) {\texttt{N}}
  (not1.in 1) node[left] {$A$}
  (not1.out) node[right] {$\overline{A}$};  
\end{circuitikz}

\subsection{AND同生死}
\begin{circuitikz}
  \draw (0,0) node[nand port] (nand1) {}
  (nand1.in 1) node[left] {$A$}
  (nand1.in 2) node[left] {$B$}
  (nand1.out) --++(.2,0)
  node[not port,anchor=in] (not1){}
  (not1.out)node[right]{$\overline{A\  \texttt{NAND}\  B} = A \cdot B$};

  \draw[shift={(7,0)}]
  (0,0) node[and port] (and1) {\texttt{AND}}
  (and1.in 1) node[left] {$A$}
  (and1.in 2) node[left] {$B$}
  (and1.out) node[right] {$A \cdot B$};
\end{circuitikz}


\subsection{NOR}
\begin{circuitikz}[american]
  \draw (0,0) node[nand port] (nand1) {}
  (nand1.in 1)--++(-.5,.5)node[not port,anchor=out](not1){}
  (nand1.in 2)--++(-.5,-.5)node[not port,anchor=out](not2){}
  (not1.in) node[left]{$A$}
  (not2.in) node[left]{$B$}
  (nand1.out)--++(.2,0)node[not port,anchor=in](not3){}
  (not3.out)node[right] {$Y = \overline{A} \cdot \overline{B}$};


  % \draw[shift={(7,0)}]
  % (0,0) node[nor port] (nor1) {\texttt{NOR}}
  % (nor1.in 1) node[left] {$A$}
  % (nor1.in 2) node[left] {$B$}
  % (nor1.out) node[right] {$Y = A + B$};
\end{circuitikz}
\end{document}
