\documentclass{article}
\usepackage[UTF8]{ctex}
\usepackage{circuitikz}
\usepackage{placeins}
\renewcommand{\floatpagefraction}{0.7}  % 降低浮动体占页面比例的要求
\renewcommand{\textfraction}{0.1}       % 降低文本占页面最小比例
\setcounter{topnumber}{3}               % 增加页面顶部浮动体数量
\setcounter{bottomnumber}{3}            % 增加页面底部浮动体数量

\tikzset{
  snot port/.style = {not port, scale=0.6}
}

\title{Code 笔记}
\author{Sunny}
\date{\today}
\begin{document}

\maketitle

\section{引言}
Turing Complete+布尔代数基础知识。
\section{基本逻辑门}
\subsection{NOT逆阴阳}
\begin{figure}[htbp]
\centering
\begin{circuitikz}
  \draw
  (0,0) node[not port] (not1) {}
  (not1.in 1) node[left] {$A$}
  (not1.out) node[right] {$\overline{A}$};  
\end{circuitikz}
\caption{NOT非门}
\end{figure}

\subsection{AND 全阳则阳}
\begin{figure}[htbp]
\centering
\begin{circuitikz}
  \draw
  (0,0) node[and port] (and1) {}
  (and1.in 1) node[left] {$A$}
  (and1.in 2) node[left] {$B$}
  (and1.out) node[right] {$A \cdot B$};
\end{circuitikz}
\caption{AND与门}
\end{figure}

\subsection{OR 全阴则阴}
\begin{figure}[htbp]
\centering
\begin{circuitikz}
  \draw
  (0,0) node[or port] (or1) {}
  (or1.in 1) node[left] {$A$}
  (or1.in 2) node[left] {$B$}
  (or1.out) node[right] {$A+B$};  
\end{circuitikz}
\caption{OR或门}
\end{figure}

\subsection{NAND 全阳转阴}
\begin{figure}[htbp]
\centering
\begin{circuitikz}
  \draw
  (0,0) node[nand port] (nand1) {}
  (nand1.in 1) node[left] {$A$}
  (nand1.in 2) node[left] {$B$}
  (nand1.out) node[right] {
    A NAND B =$\overline{A \cdot B}=\overline{A}+\overline{B}$};
\end{circuitikz}
\caption{NAND 门}
\end{figure}

\subsection{NOR 全阴转阳}
\begin{figure}[ht]
\centering  
\begin{circuitikz}
  \draw
  (0,0) node[nor port] (nor1) {}
  (nor1.in 1) node[left] {$A$}
  (nor1.in 2) node[left] {$B$}
  (nor1.out) node[right] {$\overline{A} \cdot \overline{B}=\overline{A+B}$};  
\end{circuitikz}
\caption{NOR门}
\end{figure}

\section{函数完备}
函数完备俗称万能门电路,可以用万能门能够逐步构造出其它所有门电路,只有NAND和NOR有这个性质。
\subsection{由NAND构造}
\begin{figure}[ht]
\centering
\begin{circuitikz}
  \draw (0,0) node[left]{$A$} to[short, o-] ++(.5,0)coordinate(JA)
  to[short,*-]++(.5,0) node[nand port,anchor=in 1] (nand1){}
  (JA)|-(nand1.in 2)
  (nand1.out) node[right] {$A\  \texttt{NAND}\  A = \overline{A}$};
\end{circuitikz}
\caption{NAND$\to$NOT}
\end{figure}

\begin{figure}[ht]
\centering
\begin{circuitikz}
  \draw (0,0) node[nand port] (nand1) {}
  (nand1.in 1) node[left] {$A$}
  (nand1.in 2) node[left] {$B$}
  (nand1.out) --++(.2,0)
  node[snot port,anchor=in] (not1){}
  (not1.out)node[right]{$\overline{A\  \texttt{NAND}\  B} = A \cdot B$};
\end{circuitikz}
\caption{NAND$\to$AND}
\end{figure}

\begin{figure}[ht]
\centering
\begin{circuitikz}
  \draw (0,0) node[nand port] (nand1) {}
  (nand1.in 1)--++(-.3,0)node[not port,anchor=out,scale=.4](not1){}
  (nand1.in 2)--++(-.3,0)node[not port,anchor=out,scale=.4](not2){}
  (not1.in) node[left]{$A$}
  (not2.in) node[left]{$B$}
  (nand1.out)node[right] {$A\  \texttt{OR}\  B = A+B$};  
\end{circuitikz}
\caption{NAND$\to$OR}
\end{figure}

\FloatBarrier  % 在需要的位置插入屏障
\subsection{由NOR构造}
略

\section{加法器}
基本加法涉及拆分成相加与进位两个部分,进位用AND,相加用XOR。

\subsection{XOR}

\begin{figure}[htbp]
\centering
\begin{circuitikz}
  \draw (0,0) node[xor port] (xor1) {}
  (xor1.in 1) node[left] {$A$}
  (xor1.in 2) node[left] {$B$}
  (xor1.out) node[right] {$A\oplus B$};  
\end{circuitikz}  
\caption{XOR门}
\end{figure}
\FloatBarrier  % 在需要的位置插入屏障

\subsubsection{构造XOR}
构造输入与期望真值表,发现期望与OR/NAND相差一位,放一起发现可AND后实现。

\begin{tabular}{|c|c|c|c|c|}
\hline
\textbf{A in} & \textbf{B in} & \textbf{OR out} & \textbf{NAND out} & \textbf{期望} \\
\hline
0 & 0 & 0 & 1 & 0 \\
\hline
0 & 1 & 1 & 1 & 1 \\
\hline
1 & 0 & 1 & 1 & 1 \\
\hline
1 & 1 & 1 & 0 & 0 \\
\hline
\end{tabular}

根据真值表推导出电路图
\begin{figure}[htbp]
\centering
\begin{circuitikz} %[scale=1.2, transform shape]
    \draw (0,0) node[left](A){$A$} to[short, o-]++(1,0)coordinate(JA)
    to[short,*-]++(.5,0) node[or port,anchor=in 1](or1){}
    (or1.in 2-|A.east) node[left]{$B$} to[short,o-]++(.5,0)coordinate(JB)
    to[short,*-](or1.in 2)
    (or1.out)-|+(.5,-.6)node[and port,anchor=in 1](and1){}
    (and1.in 2)|-++(-.5,-.6)node[nand port,anchor=out](nand1){}
    (JA)|-(nand1.in 1)
    (JB)|-(nand1.in 2)
    (and1.out)node[right]{$(A+B)\cdot(\overline{A\cdot B})$};
\end{circuitikz}
\caption{XOR门构造方式1}
\end{figure}

根据定义的布尔代数可推导出另一种电路图,两种方式是等价的。
\[(A+B)\cdot(\overline{A\cdot B})
=(A+B)\cdot(\overline{A}+\overline{B})
=A\overline{A}+A\overline{B}+B\overline{A}+B\overline{B}
=0+A\overline{B}+\overline{A}B+0
\]

\begin{figure}[htbp]
\centering
\begin{circuitikz}
  \draw (0,0) node[left]{$A$} to[short, o-]++(.5,0)coordinate(JA)
  (JA)--++(.5,0)node[snot port,anchor=in](nota){}
  (nota.out)-|++(.5,0) node[and port,anchor=in 1](and1){}
  (and1.out)-|++(.5,-.4)node[or port,anchor=in 1](or1){}
  (or1.in 2)|-++(-.5,-.4)node[and port,anchor=out](and2){}
  (and2.in 2)--++(-.5,0)node[snot port,anchor=out](notb){}
  (notb.in)--++(-.6,0)coordinate(JB)
  to[short,*-](JB|-and1.in 2)--(and1.in 2)
  (JB)--++(-.4,0)node[left]{$B$}node[ocirc]{}
  (JA)to[short,*-](JA|-and2.in 1)--(and2.in 1)
  (or1.out)node[right] {$\overline{A} \cdot B+A\cdot \overline{B}$};    
\end{circuitikz}
\caption{XOR门构造方式2}
\end{figure}



\subsection{半加器(Half Adder)}
\begin{circuitikz}
  \draw(0,0)node[left](A){A}to[short,o-]++(1,0)coordinate(JA)
  to[short,*-]++(.5,0)node[xor port,anchor=in 1](xor1){}
  (xor1.in 2)to[short,-*]++(-1,0)coordinate(JB)
  to[short,-o](xor1.in 2-|A.east)node[left](B){B}
  (xor1.out)node[right]{Sum}
  (xor1.east)++(0,-1.5)node[and port](and1){}
  (and1.in 1)-|(JA)
  (and1.in 2)-|(JB)
  (and1.out)node[right]{Carry};
\end{circuitikz}


\newcommand*{\ha}[2]{ %#1 node name,#2 anchor
    node[fourport,anchor=#2,scale=.65] (#1) {}
    (#1.port1) node[right]{B}
    (#1.port2) node[left]{CO}
    (#1.port3) node[left]{S}
    (#1.port4) node[right]{A}
}
简化为
\begin{circuitikz}
  \ctikzset{quadpoles/fourport/width=2};
  \draw(0,0)\ha{F0}{port4};
\end{circuitikz}


\subsection{全加器(Full Adder)}
\begin{circuitikz}  
    \ctikzset{quadpoles/fourport/width=2}
    \draw (0,0)node[left](A){A}to[short,o-]++(.5,0)
    \ha{F0}{port4}
    (F0.port3)--++(1,0)\ha{F1}{port1}
    (F0.port1)to[short,-o](F0.port1-|A.east)node[left]{B}
    (F1.port4)to[short,-o](F1.port4-|A.east)node[left]{Carry}
    (F1.port2)--++(.5,0)node[or port,anchor=in 1](or1){}
    (F0.port2)--(or1.in 2)
    (or1.out)node[right]{Carry}
    (F1.port3)--(F1.port3-|or1.out)node[right]{Sum};
\end{circuitikz}

\end{document}
