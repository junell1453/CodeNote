\documentclass{article}
\usepackage{circuitikz}
\begin{document}
% 定义半加器模块
\newcommand{\halfadder}{
    % 绘制半加器矩形框
    \draw (0,0) rectangle (3,2);

    % 在框内添加"半加器"文字
    \node at (1.5, 1) {半加器};

    % 输入端口A,B,输出端口S,CO
    \draw (0,1.5) node[right] {A} -- ++(-0.5,0); % 输入A
    \draw (0,0.5) node[right] {B} -- ++(-0.5,0); % 输入B
    \draw (3,1.5) node[left] {S} -- ++(0.5,0);  % 输出S
    \draw (3,0.5) node[left] {CO} -- ++(0.5,0); % 输出CO
}   

\begin{circuitikz}[american, scale=1, transform shape]
    \draw (0,0) node[above]{$A$} to[short, o-] ++(2,0)
    node[nand port,anchor=in 1] (nand1){}
    (nand1.in 2)--++(0,-1) 
    node[nand port,anchor=out] (nand2){}
    (nand1.out) node[right] {$Y = \overline{A}$};
\end{circuitikz}




% 使用半加器模块
\begin{circuitikz}
\halfadder
\end{circuitikz}

\end{document}
